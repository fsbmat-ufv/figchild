\documentclass[a4paper,dvipsnames]{article}
\usepackage{tikz}
\usetikzlibrary{backgrounds}
\usepackage[v2]{figchild}
\title{A new user interface for the \texttt{figchild} package}
\author{Matthias Flor\'e}
\date{2022/10/30}
\begin{document}
\maketitle
This document describes a new user interface for the \texttt{figchild} package.

The new user interface can be used with \verb|\usepackage[v2]{figchild}|. Note the package option \verb|v2|.

In this case, the macros defined in the file \verb|figchild.sty| take one optional argument which accepts a list of Ti\emph{k}Z keys.

With the old interface, a scale, color and line width have to be given as mandatory arguments. With the new interface, this looks for example as below:

\begin{tikzpicture}
\fcDaisy[
    scale=0.2,
    ForestGreen,
    line width=2pt
]
\end{tikzpicture}

The new interface allows to give different styles for different parts of the figure, as in the example below:

\begin{tikzpicture}
\fcDaisy[
    scale=0.3,
    ultra thick,
    minor A={
        Dandelion!75!black,
        fill=Dandelion
    },
    minor B=Dandelion!75!black,
    minor C={
        Brown!75!black,
        fill=Brown!50!white
    },
    minor D=black,
    minor E={
        ForestGreen,
        fill=ForestGreen!50,
        line width=2pt,
        dotted
    }
]
\end{tikzpicture}

With the old interface, a \texttt{figchild} figure was already a \verb|tikzpicture| on its own. This makes it unconvenient to make a picture with several \texttt{figchild} figures. With the new interface, a \texttt{figchild} figure is not a \verb|tikzpicture| on its own thus it is easier to make a figure such as below:

\begin{tikzpicture}[
    scale=0.19,
    background rectangle/.style={fill=ForestGreen},
    show background rectangle
]
\foreach\n in {1,...,5}{
    \fcDaisy[shift={({12*\n},{3*(-1)^\n})}]
}
\end{tikzpicture}

With the new syntax, it is also very easy to perform transformations on a figure, for example the figure below is mirrored:

\begin{tikzpicture}
\fcCloudC[
    xscale=-1,
    scale=0.5
]
\end{tikzpicture}

Below are $3$ more examples of the benefits described above. Note that for \verb|\fcChristmasTree|, the sequence in the code was changed such that the tree is drawn before the christmas balls.

\begin{tikzpicture}
\fcChristmasTree[
    minor A={
        ForestGreen,
        ultra thick,
        fill=ForestGreen!50!white
    },
    minor B={
        Brown!75!black,
        ultra thick,
        fill=Brown!50!white
    },
    minor C={
        Gray,
        thick,
        fill=Red
    }
]
\end{tikzpicture}

\begin{tikzpicture}
\fcCloudA[
    scale=0.9,
    minor A={
        SkyBlue,
        line width=2pt,
        fill=SkyBlue!25!white
    },
    minor B={
        Goldenrod,
        line width=6pt
    }
]
\end{tikzpicture}

\begin{tikzpicture}[
    background rectangle/.style={fill=MidnightBlue!75},
    show background rectangle
]
\fcCloudC[
    scale=0.5,
    SkyBlue,
    ultra thick,
    minor A=Red,
    minor B=Orange,
    minor C=Yellow,
    minor D=Green,
    minor E=Blue,
    minor F=BlueViolet,
    minor G=Violet,
    minor H={fill=SkyBlue!50!white},
    minor I={black,fill=SkyBlue!25!white}
]
\end{tikzpicture}

Below are some remarks on the code in the \verb|.sty| file:
\begin{enumerate}
\item In the code for \verb|\fcChristmasTree| was the line

\verb|\draw[draw=#2, line width=#3pt];|.

Either some drawing instruction was missing or this line can be removed.
\item In the \verb|.sty| file the macros \verb|\imagewidthh|, \verb|\imagescaleh| and \verb|\version| were defined. Hence, a user cannot use these variable names anymore because this will for example give the error

\verb|LaTeX Error: Command \version already defined.|.

This can be solved for example by putting \verb|figchild@| in front of these names, such as in the new version of the \verb|.sty| file. The \verb|\version| command does not need to be defined in the \verb|.sty| file but only in the manual.
\end{enumerate}
\end{document}